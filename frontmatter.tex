\frontmatter
\chapter{前言}
注重培养良好的思维习惯、编程风格,而不是简单的知识传授。学习Java的目的是工业化
应用,而不是玩具。

一般的,高校讲授的第一门计算机程序设计语言是C语言,因此大家在学习(自学或者教学)
Java语言时已经具有了一定的C语言基础。而目前常见的Java语言教材或者辅导材料基本上
都是零基础开始讲授Java语言程序设计的,完全无视大家的C语言基础,导致教与学活动中
的很大浪费,也不利于突出Java教学活动的重点。

以知识点为线索,以丰富的规范化实例为佐料,不同于手册,不同于百科全书的新型教材。

\section*{本书的特点}
本书通过思维导图的方式绘出了Java的各个知识点,并针对每个知识点设计了练习题,帮助
同学们更好的了解这些知识点及其在实际中的应用方式。
\section*{如何阅读本书}
边读边练

本书不强调从一开始就掌握Java的每个技术细节,而是强调“先跑起来!”,即首先动手写出
一个可以运行的Java应用程序,并理解这个应用程序为什么能够跑起来,然后再逐步扩充到
细节问题。也就是说,在实践中体味和掌握技术细节,在实践中逐步形成良好的编程风格和
思维习惯。

本教材特别强调学生主动参与的重要性。一般的,首先给出一个可以运行的简单示例,然后
通过不断提出问题的方式,让学生自己“探索”解决问题的方法,最终完成一个相当不错和有些
复杂的程序。通过这种方式将各个知识点串起来,弄清楚程序设计语言为什么这样设计和定义。
看起来一切都是学生自己完成的,这样学生的成就感很强,自学能力和自学的信心都会得到
很大程度的提高。

\section*{你适合阅读本书吗}
\section*{本书的体例}
\subsection*{本书的印刷约定}
\begin{tabular}{|l|l|l|}
    \hline
    字体 & 意义 & 示例 \\
    \hline
    \textsl{AbCd123}斜体 & 文件名、路径名、域名等 & ls -l \textsl{filename}\\
    \hline
    \textbf{AbCd123}加粗 & 在终端输入的命令等 & subaochen\_desktop\% \textbf{su} \\
    \hline
    等宽字体 & 示例代码、代码片段等 & publc class MyClass... \\
    \hline

\end{tabular}

\subsection*{类的命名方式}
本书中,类的命名遵循如下的原则:
\begin{itemize}
    \item 接口采用自然的命名方式,比如UserManager接口。
    \item 接口的实现类是在接口的名字后面增加Bean,比如UserManager接口的某个实现是DbUserManagerBean,以强化“一切都是Bean”的理念。
\end{itemize}
\section*{联系我们}
您可以通过我的博客获得最新的消息:http://dz.sdut.edu.cn/blog/subaochen,本书所有的源代码都可以从http://github.com/subaochen/java-based-on-c获得,您可以在本书的不同章节看到如何获得源代码的相关提示。

\section*{本书是如何写成的}
本书全部使用开源(Open Source)软件完成:
\begin{itemize}
    \item Lyx(\url{http://www.lyx.org}),优秀的Latex前端可视化工具,最新版本(本书写作时是2.0)配合xetex可以很好的支持中文处理。
    \item graphviz,灵活而强大的代码绘图工具。
    \item Shutter(\url{http://shutter-project.org}),Linux下面优秀的截图工具。
    \item ArgoUML(\url{http://argouml.tigris.org}),优秀的开源UML工具。
    \item Dia,优秀的开源流程图绘制工具,本书的大部分流程图和框图都是用Dia绘制的。
    \item openoffice draw,openoffice套件中的绘图工具。
\end{itemize}
\mainmatter
