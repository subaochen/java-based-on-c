\frontmatter
\chapter{前言}
注重培养良好的思维习惯、编程风格,而不是简单的知识传授。学习Java的目的是工业化应用,而不是玩具。

\section*{如何阅读本书}
边读边练

\section*{你适合阅读本书吗}
\section*{本书的体例}
\subsection*{本书的印刷约定}
\begin{tabular}{|l|l|l|}
    \hline
    字体 & 意义 & 示例 \\
    \hline
    \textsl{AbCd123}斜体 & 文件名、路径名、域名等 & ls -l \textsl{filename}\\
    \hline
    \textbf{AbCd123}加粗 & 在终端输入的命令等 & subaochen\_desktop\% \textbf{su} \\
    \hline
    等宽字体 & 示例代码、代码片段等 & publc class MyClass... \\
    \hline

\end{tabular}

\subsection*{类的命名方式}
本书中,类的命名遵循如下的原则:
\begin{itemize}
    \item 接口采用自然的命名方式,比如UserManager接口。
    \item 接口的实现类是在接口的名字后面增加Bean,比如UserManager接口的某个实现是DbUserManagerBean,以强化“一切都是Bean”的理念。
\end{itemize}
\section*{联系我们}
您可以通过我的博客获得最新的消息:http://dz.sdut.edu.cn/blog/subaochen,本书所有的源代码都可以从http://github.com/subaochen/java-based-on-c获得,您可以在本书的不同章节看到如何获得源代码的相关提示。

\section*{本书是如何写成的}
本书全部使用开源(Open Source)软件完成:
\begin{itemize}
    \item Lyx(\url{http://www.lyx.org}),优秀的Latex前端可视化工具,最新版本(本书写作时是2.0)配合xetex可以很好的支持中文处理。
    \item graphviz,灵活而强大的代码绘图工具。
    \item Shutter(\url{http://shutter-project.org}),Linux下面优秀的截图工具。
    \item ArgoUML(\url{http://argouml.tigris.org}),优秀的开源UML工具。
    \item Dia,优秀的开源流程图绘制工具,本书的大部分流程图和框图都是用Dia绘制的。
    \item openoffice draw,openoffice套件中的绘图工具。
\end{itemize}
\mainmatter
